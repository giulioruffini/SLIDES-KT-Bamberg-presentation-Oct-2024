%\section{Neurophenomenology of KT}




%%%%%%%%%%%%%%%%%%%%%%%%%%%%%%%%%%%%%%%%%%%%%%%%%%%%%%%%%%%%%%%%%%%
% \begin{frame}[label=ladila]{Altered states of \SEP}
  
%  Neurophenomenology defines a methodological strategy for integrating phenomenological and neurobiological accounts: 1P  (phenomenology---subjective) and 3P (physiology, behavior---objective) data \citep{Varela:1996aa}. \vfill
 
%  Altered states of consciousness such as psychedelics or meditation offer an interesting context to study the effect of perturbing the mechanisms of \SEP.\vfill
 
% Meditative states are associated with the global dissolution of the embodied self  \citep{Millire2018} and can  serve as unique models for  investigation of self-dissolution (disengagement of self-models).\vfill
 
%  %As an  objective measure of structured experience, we can analyze descriptive narratives in speech form through state-of-the-art computational analysis (NLP) to establish  metrics on text structure such as semantic coherence and speech disorganization index~\citep{Sanz:2021, Tagliazucchi:2022, Mota:2017}.
 
 
% \end{frame}

 
  %%%%%%%%%%%%%%%%%%%%%%%%%%%%%%%%%%%%%%%%%%%%%%%%%%%%%%
