\section{Closing}



%%%%%%%%%%%%%%%%%%%%%%%%%%%%%%%%%%%%%%%%%%%%%%
\begin{frame}[label=ladila]{Philosophy}

KT sits naturally in the context of panpsychism (`mind is everywhere', see Strawson, \cite{Goff:2019aa}), a  somewhat controversial version of the philosophy of consciousness.  \vfill

Idealism is  perhaps more rigorous philosophical background (consciousness as the fundamental entity, `mind is everything'). \vfill

 Although not necessary to explore the scientific implications of the theory, the adoption of these can itself be  motivated by simplicity and consistency criteria \citep{Symes2022-ri}.
 
\end{frame}



%%%%%%%%%%%%%%%%%%%%%%%%%%%%%%%%%%%%%%%%%%%%%%
\begin{frame}[label=ladila]{Ethics}

KT does not grant any special status to humans: all systems that capture structure from the world have structured experience.  \vfill

Pleasure/pain associated with  {\bf objective function} $O$ $\rightarrow$ morality:  natural notions of {\em good} or {\em evil} in computational terms. \vfill

E.g., we may say that Agent's $A$ is {\em evil} to Agent $B$ if the objective function   $O_A$ increases when O$_B$ decreases, that is $O_A(O_B)$ is decreasing or  $O_A'(O_B) <0$ (and viceversa for {\em good}).
 \vfill
 
% Conversely, we say that Agent $A$ is {\em good or morally right} to Agent $B$ if $O_A'(O_B) >0$. \vfill 

Synergistic behavior emerges when agents are good to each other, while mutually-destructive behavior takes place in the complementary case.  
 
\end{frame}

%%%%%%%%%%%%%%%%%%%%%%%%%%%%%%%%%
\begin{frame}[label=ladila]{Future}
Much work remains to be done! Will KT provide a unification framework for the different approaches to consciousness? IIT, GWT, FEP, DIT all seem to fit. What is the neurobiology map of agenthood?\vfill

Can we computationally\textit{ evolve agents}? Are \textit{persistent patterns} unavoidable (KT conjecture) in a computational soup if we wait long enough? Are there  types other than static (proton), and agents  (life/intelligence)? \vfill

%Can we further develop a theory bridging dynamical systems and AIT? \vfill

Can we discover the structure of reduced dynamics (hierarchical reduced manifolds) from 3P data? \vfill

Can we design better neurophenomenological methods to study \SEP?   \vfill

Can we design model-building agents mimicking life or intelligence? Is AI the next evolutionary model-building leap? Brain-to-brain communication?\vfill


\end{frame}


%%%%%%%%%%%%%%%%%%%%%%%%%%%%%%%%%
\begin{frame}[label=ladila]{Thanks}
\vfill
\begin{center}

   {\Large Thanks for your attention and curiosity!}  \vfill
   
  % {\large Thanks to  Ed and Roser for all the brainstorming.}  \vfill
    

    
    Slides available at {\small  https://github.com/giulioruffini/Ruffini-KT-Tucson-presentation-April-19-2022 }\\ $\rightarrow$~Preprint is on the way.\vspace{1.5cm} 
    
        giulio.ruffini@neuroelectrics.com, @ruffini (Twitter)  \vfill
        
\end{center}
\vfill

\end{frame}